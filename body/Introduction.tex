%%% mode: latex
%%% TeX-master: t
%%% End:

\chapter{绪论}
\label{cha:intro}

\section{研究背景与意义}
\label{sec:general intro}

金属铸造是一项拥有悠久历史的材料成形技术,能够生产具有各种尺寸、形状、数量、质量以及材料需求的金属零件\cite{}。


\subsection{**概念}



\subsection{**技术}



\section{XXX国内外研究现状}
\label{sec:requirement}

此处应就与本文相关的国内外研究概况进行全面综述,这样相关内容在后面章节中就可以点到为止,无需再大段大段地分别介绍了。

\section{存在的问题}
\label{sec:compile}

\section{研究内容}
\label{sec:xelatex}
不同学科、不同专业、不同学生学位论文的类别各异,大致可分为实验研究类、理论/算法研究类、仪器/工艺设计与研发类、综合类等。不同论文的章节结构各异,但每种类型的论文还是有其特定的格式。原则上,博士论文的主体研究内容不得少于3章,加上绪论、总结与展望,累计不得少于5章。

作为华中科技大学博士学位论文模板,本文首先给出了不同类型的研究论文典型结构供大家参考,再根据学术出版的规范化要求,说明论文写作中的细则。全文共分为5章。主要内容如下:

第1章 绪论:简要介绍论文的研究背景、国内外研究现状、存在的问题,给出全文的主要研究内容。为了让读者更容易理解全文,建议用一个文档结构图给出各章节逻辑关系。

第2章 样式使用说明:一般应介绍该章的主要内容,说明这一章做什么。

第3章 实验研究类论文:……

第4章 理论或算法类论文的结构:……

第5章 总结与展望:给出全文的主要内容及结论,总结本文的创新点,并对未来的工作进行展望。
